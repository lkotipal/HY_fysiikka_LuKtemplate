% STEP 1: Choose oneside or twoside
\documentclass[finnish,twoside,openright]{HYgradu}

\usepackage{lmodern} % Font package
\usepackage{textcomp} % Package for special symbols
\usepackage[pdftex]{color, graphicx} % For pdf output and jpg/png graphics
\usepackage[pdftex, plainpages=false]{hyperref} % For hyperlinks and pdf metadata
\usepackage{fancyhdr} % For nicer page headers
\usepackage{tikz} % For making vector graphics (hard to learn but powerful)
%\usepackage{wrapfig} % For nice text-wrapping figures (use at own discretion)
\usepackage{amsmath, amssymb} % For better math
%\usepackage[square]{natbib} % For bibliography
\usepackage[footnotesize,bf]{caption} % For more control over figure captions
\usepackage{blindtext}
\usepackage{titlesec}
\usepackage[titletoc]{appendix}

\onehalfspacing %line spacing
%\singlespacing
%\doublespacing

%\fussy 
\sloppy % sloppy and fussy commands can be used to avoid overlong text lines

% STEP 2:
% Set up all the information for the title page and the abstract form.
% Replace parameters with your information.
\title{Opinn\"aytety\"opohja}
\author{Roope Halonen ja Tomi Vainio}
\date{\today}
%\level{Bachelor's thesis}
\level{Kandidaatintutkielma}
\subject{P\"a\"aaine/Opintosuunta}
%\subject{Your Field}
\faculty{Matemaattis-luonnontieteellinen tiedekunta}
%\faculty{Faculty of Whatever}
\programme{Fysikaalisten tieteiden kandidaattiohjelma}
\address{PL 64 (Gustaf H\"allstr\"omin katu 2a)\\00014 Helsingin yliopisto}
\prof{professori/dosentti/jne Etunimi Sukunimi}
\censors{Etunimi Sukunimi}{Etunimi Sukunimi}{}
\keywords{\LaTeX}
\depositeplace{}
\additionalinformation{}
\classification{}

% if you want to quote someone special. You can comment this line and there will be nothing on the document.
%\quoting{Bachelor's degrees make pretty good placemats if you get them laminated.}{Jeph Jacques} 

% OPTIONAL STEP: Set up properties and metadata for the pdf file that pdfLaTeX makes.
% But you don't really need to do this unless you want to.
\hypersetup{
    bookmarks=true,         % show bookmarks bar first?
    unicode=true,           % to show non-Latin characters in Acrobat’s bookmarks
    pdftoolbar=true,        % show Acrobat’s toolbar?
    pdfmenubar=true,        % show Acrobat’s menu?
    pdffitwindow=false,     % window fit to page when opened
    pdfstartview={FitH},    % fits the width of the page to the window
    pdftitle={},            % title
    pdfauthor={},           % author
    pdfsubject={},          % subject of the document
    pdfcreator={},          % creator of the document
    pdfproducer={pdfLaTeX}, % producer of the document
    pdfkeywords={something} {something else}, % list of keywords for
    pdfnewwindow=true,      % links in new window
    colorlinks=true,        % false: boxed links; true: colored links
    linkcolor=black,        % color of internal links
    citecolor=black,        % color of links to bibliography
    filecolor=magenta,      % color of file links
    urlcolor=cyan           % color of external links
}

\begin{document}
% Generate title page.
\maketitle

% STEP 3:
% Write your abstract (of course you really do this last).
% You can make several abstract pages (if you want it in different languages),
% but you should also then redefine some of the above parameters in the proper
% language as well, in between the abstract definitions.

\begin{abstract}
Kirjoita tiivistelm\"a\"an lyhyt, enint\"a\"an 250 sanan yhteenveto ty\"ost\"asi: mit\"a olet tutkinut, millaisia menetelmi\"a olet k\"aytt\"anyt, millaisia tuloksia sait ja millaisia johtop\"a\"at\"oksi\"a niiden perusteella voi tehd\"a.
\end{abstract}

% Place ToC
\mytableofcontents

\mynomenclature

% -----------------------------------------------------------------------------------
% STEP 4: Write the thesis.
% Your actual text starts here. You shouldn't mess with the code above the line except
% to change the parameters. Removing the abstract and ToC commands will mess up stuff.
\chapter{Johdanto}

Fysiikan laitoksen opinn\"aytety\"opohja k\"aytt\"a\"a omaa dokumenttiluokkaansa \texttt{HYgradu.cls}, joka m\"a\"aritt\"a\"a dokumentin tyylin ja luo automaattisesti kansi- ja tiivistelm\"asivun annettujen parametrien mukaan. Opiskelijalla on mahdollisuus muuttaa dokumenttiluokan asetuksia halutessaan, mutta on suositeltavaa, ett\"a yleist\"a ulkoasua noudatettaisiin etenkin kansilehden ja tiivistelm\"an kohdalla.

\TeX-tiedoston alussa on lista kohtia, joihin tulee t\"aytt\"a\"a tietoja esimerkiksi nimi, ty\"on otsikko jne\footnote{Ty\"on sivum\"a\"ar\"a lasketaan suoraan dokumentista siten ett\"a laskenta alkaa ensimm\"aisest\"a luvusta.}. \LaTeX-ohjelmisto luo dokumentin n\"aiden tietojen perusteella automaattisesti.

\section{Dokumentin k\"a\"ant\"aminen}
\label{sec:compile}

Opinn\"aytety\"o tulee k\"a\"ant\"a\"a pdf-tiedostoksi \emph{makefile}a k\"aytt\"am\"all\"a, koska dokumenttiin generoidaan automaattisesti l\"ahdeluettelo ja mahdollisesti symboliluettelo\footnote{N\"aiden luetteloiden luominen vaatii dokumentin k\"a\"ant\"amist\"a lis\"aksi \textsc{B\kern-0.1emi\kern-0.017emb}\kern-0.15em\TeX- ja makeindex-ty\"okaluilla.}.

Tiedostossa \texttt{makefile} m\"a\"aritell\"a\"an \TeX-tiedoston ja l\"ahdeluettelotiedoston nimet:
\begin{verbatim}
name=HYluk_template
bibfile=bibliography.bib
...
\end{verbatim}
Ty\"on voi k\"a\"ant\"a\"a ajamalla komento \texttt{make} komentorivilt\"a, jolloin ty\"o k\"a\"annet\"a\"an useita kertoja siten ett\"a dokumentin kaikki luettelot ovat p\"aivittyneet.

Usein pdf-tiedostoa halutaan p\"aivitt\"a\"a vain pienten muutosten j\"alkeen, jos uusia viittauksia tai symboleita ei ole luotu. Etenkin hyvin pitk\"an ja paljon kuvia sis\"alt\"av\"an \TeX-tiedoston k\"a\"ant\"aminen \texttt{make}-komennolla saattaa vied\"a huomattavasti aikaa.
T\"all\"oin voi k\"aytt\"a\"a komentoa \texttt{make simple}, joka k\"a\"ant\"a\"a dokumentin vain yhden kerran kolmen sijaan. 

Jos dokumentin kielt\"a tai viittaustyyli\"a muutetaan tulee hakemistosta poistaa aputiedostot (\texttt{.aux}, \texttt{.blg} jne.). Poistamisen voi suorittaa automaattisesti ajamalla komento \texttt{make clean}.

\chapter{Ulkoasu}

\section{Kieli}

Dokumentin kieleksi voi valita suomen (\texttt{finnish}), ruotsin (\texttt{swedish}) ja englannin (\texttt{english}). Dokumentin oletusotsikot (\emph{Kirjallisuutta}, \emph{Liite}, \emph{Symboliluettelo}...) vaihtuvat automaattisesti vastaaman valittua kielt\"a\footnote{Muista putsata aputiedostot (luku \ref{sec:compile}).}. 

\section{Merkist\"okoodaus}
 
Riippuen k\"aytetyst\"a tietokonej\"arjestelm\"ast\"a ja valitusta kielest\"a, on t\"arke\"a\"a ett\"a dokumenttiin kirjoitetut merkit ja j\"arjestelm\"an merkist\"okoodaus vastaavat toisiaan. Vaikka \"a\"akk\"oset\footnote{T\"ass\"a esimerkki tiedostossa \"a:t ja \"o:t ovat kirjoitettu erikoismerkkein\"a k\"aytt\"aen komentoja \texttt{\textbackslash"a} ja \texttt{\textbackslash"o}.} n\"aytt\"av\"at tekstieditorissa oikeilta, tietokoneen k\"a\"ant\"aess\"a \TeX-tiedostoa se saattaa tulkita merkit v\"a\"arin. Jotta k\"a\"ant\"aminen olisi mahdollista on \TeX-tiedostossa k\"aytettyjen merkkien vastattava valittua merkist\"okoodausta. 

Kansainv\"aliseksi merkist\"okoodausstandardiksi on valikoitunut ns. Latin-9 (ISO 8859-15), joka on my\"os t\"am\"an opinn\"aytepohjan oletus. Monet tietokonej\"arjestelm\"at k\"aytt\"av\"at kuitenkin oletuksena UTF-8 -koodausta, joka saattaa aiheuttaa ongelmia dokumenttia k\"a\"ant\"aess\"a. T\"am\"an ongelman voi korjata asettamalla UTF-8 -koodauksen oletukseksi lis\"a\"am\"all\"a \texttt{utf}-komennon dokumenttiluokan ehdoksi seuraavasti:
\begin{verbatim}
\documentclass[finnish,twoside,openright,utf]{HYgradu}
\end{verbatim}
Merkist\"okoodauksen voi tarvittaessa vaihtaa my\"os tietokoneen tai k\"aytetyn tekstieditorin asetuksista.
 
\section{Kaksipuolinen vai ei?}

Painetun opinn\"aytety\"on tulisi olla kaksipuolinen, jolloin dokumentin asetukset m\"a\"aritet\"a\"an seuraavasti:
\begin{verbatim}
\documentclass[finnish,twoside,openright]{HYgradu}
\end{verbatim}
Hyv\"a k\"ayt\"ant\"o kirjojen teossa on aloittaa jokainen luku parittomalta sivulta eli oikealta vaikka edellinen luku olisi loppunut my\"os parittomalle sivulle (v\"aliin j\"a\"av\"a sivu on t\"aysin blanko). T\"am\"a tapahtuu automaattisesti k\"aytt\"am\"all\"a optiota \texttt{twoside}. Paperille tulostetuissa sel\"ast\"a\"an nidottavissa  tutkielmissa ns. niskanpuolinen marginaali on 0.5 cm leve\"ampi, koska kirjansidonta (etenkin niiteill\"a) sy\"o hieman marginaalia. 
Ty\"on voi my\"os latoa yksipuolisesti k\"aytt\"am\"all\"a optiota \texttt{oneside}. T\"all\"oin my\"os mahdolliset tyhj\"at sivut poistuvat dokumentista ja sit\"a on helpompi lukea n\"aytt\"op\"a\"atteelt\"a. Vain elektronisena versiona luettavissa tutkielmissa molemmat marginaalit ovat yht\"a leveit\"a.

\section{Kirjasinkoko ja riviv\"ali}

Dokumentin oletuskirjasinkoko on \texttt{12pt}, mutta asetuksiin voi vaihtaa kirjasinkokoon \texttt{10pt} tai \texttt{11pt}.

Riviv\"alin korkeutta voi muuttaa komennoilla \texttt{\textbackslash onehalfspacing}, \texttt{\textbackslash singlespacing}
 ja \texttt{\textbackslash doublespacing}.

\chapter{Kaavat ja symbolit}

Ideaalikaasun tilanyht\"al\"o on
\begin{equation}
\label{eq:ideal}
pV=NkT,
\end{equation}
miss\"a $p$ on kaasun paine, $V$ kaasun tilavuus, $n$ kaasun ainem\"a\"ar\"a, $T$ l\"amp\"otila ja $k$ on Boltzmannin vakio.

Suureita merkitsev\"at symbolit voi lis\"at\"a suoraan symboliluetteloon k\"aytt\"am\"all\"a komentoa
\begin{verbatim}
\nomenclature{<symboli>}{<selitys> \nomunit{<mahdollinen yksikko>}}
\end{verbatim}
Esimerkiksi paineen symboli saadaan symboliluetteloon komennolla
\begin{verbatim}
\nomenclature{$p$}{Paine \nomunit{Pa}}
\end{verbatim}
\nomenclature{$p$}{Paine \nomunit{Pa}}
\nomenclature{$V$}{Tilavuus \nomunit{m$^3$}}
\nomenclature{$n$}{Ainem\"a\"ar\"a \nomunit{mol}}
\nomenclature{$T$}{L\"amp\"otila \nomunit{K}}
\nomenclature{$k$}{Boltzmannin vakio \nomunit{1.38$\times10^{-23}$ J/K}}
Tietokone j\"arjest\"a\"a symbolit automaattisesti aakkosj\"arjestykseen, siten ett\"a isot kirjaimet ovat ennen pieni\"a ja kreikkalaiset kirjaimet ennen latinalaisia.

\begin{center}
\framebox[13.5cm]{
\begin{minipage}{13cm}
\textcolor{red}{\bf Huom!} On luettavuuden kannalta t\"arke\"a\"a, ett\"a kaavoissa suureiden symbolit ovat kirjoitettu kursiivilla (\emph{italics}), mutta yksik\"ot ja tunnukset kirjoitetaan pystykirjaimin (roman). Esim. ``Boltzmannin vakio kertoo kaasuvakion $R$ suhteen Avogadron vakioon $N_{\rm A}$...'' eik\"a ``...$N_{\color{red}{A}}$...'', koska alaindeksi A viittaa herra Amedeo Avogadroon eik\"a esimerkiksi pinta-alaan $A$.
\nomenclature{$N_{\rm A}$}{Avogadron vakio \nomunit{6.02$\times10^{23}$ mol$^{-1}$}}
\nomenclature{$A$}{Pinta-ala \nomunit{m$^2$}}
\nomenclature{$R$}{Kaasuvakio  \nomunit{8.31 J/K/mol}}
\end{minipage}
}
\end{center}

\chapter{Kuvat ja taulukot}

\section{Kuvat}
Kuvien julkaisemiseen kannattaa k\"aytt\"a\"a  jotain PostScript-tiedostomuotoa, esimerkiksi pdf- ja eps-formaatit ovat suositeltavia, koska kyseiset tiedostot sis\"alt\"av\"at tietoja kirjasimista, muotoilusta ja tarkkuudesta.

\begin{figure}[h!] 
% [h!] k\"askee ohjelman sijoittaa kuva juuri t\"ah\"an kohtaan teksti\"a.
% Vastaavasti [t] sijoittaa sivun yl\"alaitaan, [b] sijoittaa sivun alalaitaan.
\centering %sijoittaa kuvan keskelle
\includegraphics[width=0.3\textwidth]{sinetti.png}
\caption{Fysiikan laitoksen logo, t\"ass\"a vain mannekiinina kuvan asettelusta.}
\label{fig:sinetti}
% Netist\"a l\"oytyy useita paketteja jotka auttavat kuvan tai kuvatekstin sijoittelussa.
\end{figure}

\section{Taulukot}

Tuloksia esitelt\"aess\"a kannattaa tekstin lis\"aksi k\"aytt\"a\"a hyv\"aksi kuvia ja usein my\"os taulukoita. Kuvateksti tulee kuvan alapuolelle, taulukkoteksti taas taulukon  yl\"apuolelle.

Kuvien ja taulukoiden tekstit ovat astetta pienemm\"all\"a fontilla ja nimi\"o on lihavoitu.

\begin{table*}
\centering
\caption{T\"arkeimm\"at tulokset}
\label{tab:symbols}
\begin{tabular}{l||l c r} % Pystyviivoja voi laittaa niin monta kuin haluaa, eiv\"atk\"a ne ole pakollisia taulukon toiminnalle. 
% l-merkint\"a (left) sijoittaa pylv\"aik\"on solujen arvot vasemalle
% c-merkint\"a (center) sijoittaa pylv\"aik\"on solujen arvot keskelle
% r-merkint\"a (right) sijoittaa pylv\"aik\"on solujen arvot oikealle
Koe & 1 & 2 & 3 \\ 
\hline \hline % \hline luo vaakasuunnassa viivan rivien v\"aliin. Useampi \hline luo useita viivoja kyseisien rivien v\"aliin.\"a
$A$ & 2.5 & 4.7 & -11 \\
$B$ & 8.0 & -3.7 & 12.6 \\
$A+B$ & 10.5 & 1.0 & 1.6 \\
\hline
%
\end{tabular}
\end{table*}

\chapter{Viittaukset}

Dokumentin viittaustyylein\"a ovat ``numeroviitteet'' \texttt{unsrt}, jolloin viitteet tulevat viitelistaan viittausj\"arjestyksess\"a,  ja  ``tekij\"an nimi nimi ja vuosi'' -tyyli \texttt{apalike}, jolloin viitteet tulevat viitelistaan ensimm\"aisen tekij\"an sukunimen mukaisessa aakkosj\"arjestyksess\"a. Kysy ohjaajaltasi kumpaa viitaustyyli\"a h\"an suosittelee k\"aytett\"av\"aksi.

Viittauksiin k\"aytet\"a\"an erillist\"a .bib-tiedostoa. T\"ass\"a dokumentissa se on \texttt{bibliography.bib} ja n\"aytt\"a\"a t\"alt\"a:
\begin{verbatim}
@article{einstein,
    author =       "Albert Einstein",
    title =        "{Zur Elektrodynamik bewegter K{\"o}rper}. ({German})
        [{On} the electrodynamics of moving bodies]",
    journal =      "Annalen der Physik",
    volume =       "322",
    number =       "10",
    pages =        "891--921",
    year =         "1905",
    DOI =          "http://dx.doi.org/10.1002/andp.19053221004"
}
 
@book{latexcompanion,
    author    = "Michel Goossens and Frank Mittelbach and Alexander Samarin",
    title     = "The \LaTeX\ Companion",
    year      = "1993",
    publisher = "Addison-Wesley",
    address   = "Reading, Massachusetts"
}
 
@misc{knuthwebsite,
    author    = "Donald Knuth",
    title     = "Knuth: Computers and Typesetting",
    url       = "http://www-cs-faculty.stanford.edu/\~{}uno/abcde.html"
}
\end{verbatim}

Viittaus tehd\"a\"an komennolla \texttt{\textbackslash cite\{einstein\}}. Esim. 
\cite{einstein}, \cite{latexcompanion} ja \cite{knuthwebsite}\footnote{Viimeisest\"a viitteest\"a puuttuu vuosiluku, koska sit\"a ei ole m\"a\"aritelty .bib-tiedostossa.}.

\chapter{ Opinn\"aytety\"on rakenne}

Tutustu Kielijelppiin http://www.kielijelppi.fi, joka Helsingin  yliopiston kielikeskuksen yll\"apit\"am\"a verkko-opas hyv\"a\"an akateemiseen kielenk\"aytt\"o\"on suomeksi ja ruotsiksi.  

\section{Johdanto}
Johdannossa kuvataan ty\"on tausta, sen liittyminen muuhun tutkimukseen, k\"aytetty tarkastelutapa sek\"a ty\"on tavoitteet. Ajatuksena on siis kertoa, mihin tieteellisiin kysymyksiin haetaan vastausta ja mill\"a tavoin.

Johdantoon voidaan sis\"allytt\"a\"a historiikki ja kirjallisuuskatsaus, jos t\"am\"a on tarpeen asian ymm\"art\"amiseksi. Jos kirjallisuuskatsauksesta tulee pitk\"a, voi kuitenkin olla mielek\"ast\"a erottaa se omaksi luvukseen.

\section{Tekstiluvut}

Tutkielman lukujako riippuu sen sis\"all\"ost\"a. Mieti, mik\"a on oman ty\"osi kannalta tarkoituksenmukaista.

Jos ty\"on ymm\"art\"aminen vaatii enemm\"an teoreettista tietoa kuin  opintojen samassa vaiheessa olevalla mutta ty\"on aiheeseen perehtym\"att\"om\"all\"a   fysiikan opiskelijalla voi olettaa olevan, teoreettinen tausta yht\"al\"oineen on syyt\"a kuvata omassa luvussa (esim. ``Teoriaa'').
Kielenk\"ayt\"oiss\"a tulee pit\"ayty\"a  asiallisessa kirjakieless\"a puhekielisi\"a ilmaisuja v\"altt\"aen.

Jos ty\"oh\"on sis\"altyy omia mittauksia tai data-analyysi\"a, kuvaa ensin k\"aytt\"am\"asi  menetelm\"at  ja/tai aineisto (luku ``Aineisto ja/tai  menetelm\"at'') ja sen j\"alkeen saamasi tulokset (yhdess\"a tai useammassa luvussa).

Puhtaissa kirjallisuust\"oiss\"a luvut ``Aineisto ja menetelm\"at'' ja ``Tulokset'' eiv\"at ole tarpeen. Lukujako valitaan niin, ett\"a  asiat  voidaan esitell\"a mahdollisimman loogisessa j\"arjestyksess\"a.

\section{Teoriaa}
%
% T\"ass\"a esimerkit yht\"al\"ost\"a sek\"a l\"ahdeviittauksista.
%
Mahdollisessa teorialuvussa voidaan muun muassa esitell\"a ty\"oss\"a k\"aytett\"avi\"a yht\"al\"oit\"a. Yleisen omegayht\"al\"on
\begin{equation}
\label{eq:omega}
L(\omega)=F_V + F_T + F_V + F_Q + F_A
\end{equation}
avulla voidaan arvioida ilmakeh\"an pystyliikkeit\"a. Einsteinille  yht\"al\"o (\ref{eq:omega}) ei kuitenkaan ollut tuttu (\cite{einstein}), eiv\"atk\"a my\"osk\"a\"an \cite{latexcompanion} tai \cite{knuthwebsite} k\"asittele sit\"a. 
\nomenclature{$L(\omega)$}{Omega \nomunit{yksikk\"o}}

\section{Tulokset}
%
% T\"ass\"a esimerkit kuvien ja taulukoiden k\"ayt\"ost\"a.
%
Tuloksia esitelt\"aess\"a kannattaa tekstin lis\"aksi k\"aytt\"a\"a hyv\"aksi kuvia ja usein my\"os taulukoita. Muista ett\"a kuvateksti tulee kuvan alapuolelle, taulukkoteksti taas taulukon yl\"apuolelle.

\section{Johtop\"a\"at\"okset}

Kertaa lyhyesti ty\"on tavoitteet ja k\"aytt\"am\"asi menetelm\"at. Tee yhteenveto t\"arkeimmist\"a tuloksista. Pohdi my\"os tulosten merkityst\"a ja mahdollisia jatkotutkimustarpeita.

\begin{appendices}
\myappendixtitle

\chapter{Liitteet}
Liitteiss\"a voi esitell\"a esimerkiksi ty\"oss\"a k\"aytettyj\"a tietokonekoodeja:
\begin{verbatim}
#!/bin/bash          
text="Hello World!"
echo $text
\end{verbatim}

\end{appendices}

% STEP 5:
% Uncomment the following lines and set your .bib file and desired bibliography style
% to make a bibliography with BibTeX.
% Alternatively you can use the thebibliography environment if you want to add all
% references by hand.

\cleardoublepage %fixes the position of bibliography in bookmarks
\phantomsection

\addcontentsline{toc}{chapter}{\bibname} % This lines adds the bibliography to the ToC
%\bibliographystyle{unsrt} % numbering 
\bibliographystyle{apalike} % name, year
\bibliography{bibliography.bib}

\end{document}
