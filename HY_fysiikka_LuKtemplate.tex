% STEP 1: Choose oneside or twoside
\documentclass[finnish,twoside,openright,utf]{HYgradu}

\usepackage{lmodern} % Font package
\usepackage{textcomp} % Package for special symbols
\usepackage[pdftex]{color, graphicx} % For pdf output and jpg/png graphics
\usepackage[pdftex, plainpages=false]{hyperref} % For hyperlinks and pdf metadata
\usepackage{fancyhdr} % For nicer page headers
\usepackage{tikz} % For making vector graphics (hard to learn but powerful)
%\usepackage{wrapfig} % For nice text-wrapping figures (use at own discretion)
\usepackage{amsmath, amssymb} % For better math
%\usepackage[square]{natbib} % For bibliography
\usepackage[footnotesize,bf]{caption} % For more control over figure captions
\usepackage{blindtext}
\usepackage{titlesec}
\usepackage[titletoc]{appendix}
\usepackage{physics} % Good physics macros

\onehalfspacing %line spacing
%\singlespacing
%\doublespacing

%\fussy 
\sloppy % sloppy and fussy commands can be used to avoid overlong text lines

% STEP 2:
% Set up all the information for the title page and the abstract form.
% Replace parameters with your information.
\title{Opinnäytetyöpohja}
\author{Roope Halonen ja Tomi Vainio}
\date{\today}
%\level{Bachelor's thesis}
\level{Kandidaatintutkielma}
\subject{Pääaine/Opintosuunta}
%\subject{Your Field}
\faculty{Matemaattis-luonnontieteellinen tiedekunta}
%\faculty{Faculty of Whatever}
\programme{Fysikaalisten tieteiden kandidaattiohjelma}
\address{PL 64 (Gustaf Hällströmin katu 2a)\\00014 Helsingin yliopisto}
\prof{professori/dosentti/jne Etunimi Sukunimi}
\censors{Etunimi Sukunimi}{Etunimi Sukunimi}{}
\keywords{\LaTeX}
\depositeplace{}
\additionalinformation{}
\classification{}

% if you want to quote someone special. You can comment this line and there will be nothing on the document.
%\quoting{Bachelor's degrees make pretty good placemats if you get them laminated.}{Jeph Jacques} 

% OPTIONAL STEP: Set up properties and metadata for the pdf file that pdfLaTeX makes.
% But you don't really need to do this unless you want to.
\hypersetup{
    bookmarks=true,         % show bookmarks bar first?
    unicode=true,           % to show non-Latin characters in Acrobat’s bookmarks
    pdftoolbar=true,        % show Acrobat’s toolbar?
    pdfmenubar=true,        % show Acrobat’s menu?
    pdffitwindow=false,     % window fit to page when opened
    pdfstartview={FitH},    % fits the width of the page to the window
    pdftitle={},            % title
    pdfauthor={},           % author
    pdfsubject={},          % subject of the document
    pdfcreator={},          % creator of the document
    pdfproducer={pdfLaTeX}, % producer of the document
    pdfkeywords={something} {something else}, % list of keywords for
    pdfnewwindow=true,      % links in new window
    colorlinks=true,        % false: boxed links; true: colored links
    linkcolor=black,        % color of internal links
    citecolor=black,        % color of links to bibliography
    filecolor=magenta,      % color of file links
    urlcolor=cyan           % color of external links
}

\begin{document}
% Generate title page.
\maketitle

% STEP 3:
% Write your abstract (of course you really do this last).
% You can make several abstract pages (if you want it in different languages),
% but you should also then redefine some of the above parameters in the proper
% language as well, in between the abstract definitions.

\begin{abstract}
Kirjoita tiivistelmään lyhyt, enintään 250 sanan yhteenveto työstäsi: mitä olet tutkinut, millaisia menetelmiä olet käyttänyt, millaisia tuloksia sait ja millaisia johtopäätöksiä niiden perusteella voi tehdä.
\end{abstract}

% Place ToC
\mytableofcontents

\mynomenclature

% -----------------------------------------------------------------------------------
% STEP 4: Write the thesis.
% Your actual text starts here. You shouldn't mess with the code above the line except
% to change the parameters. Removing the abstract and ToC commands will mess up stuff.
\chapter{Johdanto}

Fysiikan osaston opinnäytetyöpohja käyttää omaa dokumenttiluokkaansa \texttt{HYgradu.cls}, joka määrittää dokumentin tyylin ja luo automaattisesti kansi- ja tiivistelmäsivun annettujen parametrien mukaan. Opiskelijalla on mahdollisuus muuttaa dokumenttiluokan asetuksia halutessaan, mutta on suositeltavaa, että yleistä ulkoasua noudatettaisiin etenkin kansilehden ja tiivistelmän kohdalla.

\TeX-tiedoston alussa on lista kohtia, joihin tulee täyttää tietoja esimerkiksi nimi, työn otsikko jne\footnote{Työn sivumäärä lasketaan suoraan dokumentista siten että laskenta alkaa ensimmäisestä luvusta.}. \LaTeX-ohjelmisto luo dokumentin näiden tietojen perusteella automaattisesti.

\section{Dokumentin kääntäminen}
\label{sec:compile}

Opinnäytetyö tulee kääntää pdf-tiedostoksi \emph{makefile}a tai \emph{latexmk}:ta käyttämällä, koska dokumenttiin generoidaan automaattisesti lähdeluettelo ja mahdollisesti symboliluettelo\footnote{Näiden luetteloiden luominen vaatii dokumentin kääntämistä lisäksi \textsc{B\kern-0.1emi\kern-0.017emb}\kern-0.15em\TeX- ja makeindex-työkaluilla.}.

Tiedostossa \texttt{makefile} määritellään \TeX-tiedoston ja lähdeluettelotiedoston nimet:
\begin{verbatim}
name=HYluk_template
bibfile=bibliography.bib
...
\end{verbatim}
Työn voi kääntää ajamalla komento \texttt{make} komentoriviltä, jolloin työ käännetään useita kertoja siten että dokumentin kaikki luettelot ovat päivittyneet.

Usein pdf-tiedostoa halutaan päivittää vain pienten muutosten jälkeen, jos uusia viittauksia tai symboleita ei ole luotu. Etenkin hyvin pitkän ja paljon kuvia sisältävän \TeX-tiedoston kääntäminen \texttt{make}-komennolla saattaa viedä huomattavasti aikaa.
Tällöin voi käyttää komentoa \texttt{make simple}, joka kääntää dokumentin vain yhden kerran kolmen sijaan. 

Jos dokumentin kieltä tai viittaustyyliä muutetaan tulee hakemistosta poistaa aputiedostot (\texttt{.aux}, \texttt{.blg} jne.). Poistamisen voi suorittaa automaattisesti ajamalla komento \texttt{make clean}.

Vaihtoehtoisesti mikäli asennettuna on sekä latexmk että Perl, voit kääntää ajamalla \texttt{latexmk}. Aputiedostot poistetaan ajamalla \texttt{latexmk -c}, ja kaikki kääntäjän luomat tiedostot \texttt{latexmk -C}.

\chapter{Ulkoasu}

\section{Kieli}

Dokumentin kieleksi voi valita suomen (\texttt{finnish}), ruotsin (\texttt{swedish}) ja englannin (\texttt{english}). Dokumentin oletusotsikot (\emph{Kirjallisuutta}, \emph{Liite}, \emph{Symboliluettelo}...) vaihtuvat automaattisesti vastaaman valittua kieltä\footnote{Muista putsata aputiedostot (luku \ref{sec:compile}).}. 

\section{Merkistökoodaus}
 
Riippuen käytetystä tietokonejärjestelmästä ja valitusta kielestä, on tärkeää että dokumenttiin kirjoitetut merkit ja järjestelmän merkistökoodaus vastaavat toisiaan. Vaikka ääkköset näyttävät tekstieditorissa oikeilta, tietokoneen kääntäessä \TeX-tiedostoa se saattaa tulkita merkit väärin. Jotta kääntäminen olisi mahdollista on \TeX-tiedostossa käytettyjen merkkien vastattava valittua merkistökoodausta. 

Mukavuussyistä tämä dokumentti käyttää UTF-8 -koodausta, mikä on useimmin oletuksena. Mikäli haluat jostain syystä käyttää Latin-9 -koodausta, poista komento \texttt{utf} luokan ehdoista:
\begin{verbatim}
\documentclass[finnish,twoside,openright]{HYgradu}
\end{verbatim}
Merkistökoodauksen voi tarvittaessa vaihtaa myös tietokoneen tai käytetyn tekstieditorin asetuksista.
 
\section{Kaksipuolinen vai ei?}

Painetun opinnäytetyön tulisi olla kaksipuolinen, jolloin dokumentin asetukset määritetään seuraavasti:
\begin{verbatim}
\documentclass[finnish,twoside,openright]{HYgradu}
\end{verbatim}
Hyvä käytäntö kirjojen teossa on aloittaa jokainen luku parittomalta sivulta eli oikealta vaikka edellinen luku olisi loppunut myös parittomalle sivulle (väliin jäävä sivu on täysin blanko). Tämä tapahtuu automaattisesti käyttämällä optiota \texttt{twoside}. Paperille tulostetuissa selästään nidottavissa  tutkielmissa ns. niskanpuolinen marginaali on 0.5 cm leveämpi, koska kirjansidonta (etenkin niiteillä) syö hieman marginaalia. 
Työn voi myös latoa yksipuolisesti käyttämällä optiota \texttt{oneside}. Tällöin myös mahdolliset tyhjät sivut poistuvat dokumentista ja sitä on helpompi lukea näyttöpäätteeltä. Vain elektronisena versiona luettavissa tutkielmissa molemmat marginaalit ovat yhtä leveitä.

\section{Kirjasinkoko ja riviväli}

Dokumentin oletuskirjasinkoko on \texttt{12pt}, mutta asetuksiin voi vaihtaa kirjasinkokoon \texttt{10pt} tai \texttt{11pt}.

Rivivälin korkeutta voi muuttaa komennoilla \texttt{\textbackslash onehalfspacing}, \texttt{\textbackslash singlespacing}
 ja \texttt{\textbackslash doublespacing}.

\chapter{Kaavat ja symbolit}

Ideaalikaasun tilanyhtälö on
\begin{equation}
\label{eq:ideal}
pV=NkT,
\end{equation}
missä $p$ on kaasun paine, $V$ kaasun tilavuus, $n$ kaasun ainemäärä, $T$ lämpötila ja $k$ on Boltzmannin vakio.

Suureita merkitsevät symbolit voi lisätä suoraan symboliluetteloon käyttämällä komentoa
\begin{verbatim}
\nomenclature{<symboli>}{<selitys> \nomunit{<mahdollinen yksikko>}}
\end{verbatim}
Esimerkiksi paineen symboli saadaan symboliluetteloon komennolla
\begin{verbatim}
\nomenclature{$p$}{Paine \nomunit{Pa}}
\end{verbatim}
\nomenclature{$p$}{Paine \nomunit{Pa}}
\nomenclature{$V$}{Tilavuus \nomunit{m$^3$}}
\nomenclature{$n$}{Ainemäärä \nomunit{mol}}
\nomenclature{$T$}{Lämpötila \nomunit{K}}
\nomenclature{$k$}{Boltzmannin vakio \nomunit{1.38$\times10^{-23}$ J/K}}
Tietokone järjestää symbolit automaattisesti aakkosjärjestykseen, siten että isot kirjaimet ovat ennen pieniä ja kreikkalaiset kirjaimet ennen latinalaisia.

\begin{center}
\framebox[13.5cm]{
\begin{minipage}{13cm}
\textcolor{red}{\bf Huom!} On luettavuuden kannalta tärkeää, että kaavoissa suureiden symbolit ovat kirjoitettu kursiivilla (\emph{italics}), mutta yksiköt ja tunnukset kirjoitetaan pystykirjaimin (roman). Esim. ``Boltzmannin vakio kertoo kaasuvakion $R$ suhteen Avogadron vakioon $N_{\rm A}$...'' eikä ``...$N_{\color{red}{A}}$...'', koska alaindeksi A viittaa herra Amedeo Avogadroon eikä esimerkiksi pinta-alaan $A$.
\nomenclature{$N_{\rm A}$}{Avogadron vakio \nomunit{6.02$\times10^{23}$ mol$^{-1}$}}
\nomenclature{$A$}{Pinta-ala \nomunit{m$^2$}}
\nomenclature{$R$}{Kaasuvakio  \nomunit{8.31 J/K/mol}}
\end{minipage}
}
\end{center}

\chapter{Kuvat ja taulukot}

\section{Kuvat}
Kuvien julkaisemiseen kannattaa käyttää  jotain PostScript-tiedostomuotoa, esimerkiksi pdf- ja eps-formaatit ovat suositeltavia, koska kyseiset tiedostot sisältävät tietoja kirjasimista, muotoilusta ja tarkkuudesta.

\begin{figure}[h!] 
% [h!] käskee ohjelman sijoittaa kuva juuri tähän kohtaan tekstiä.
% Vastaavasti [t] sijoittaa sivun ylälaitaan, [b] sijoittaa sivun alalaitaan.
\centering %sijoittaa kuvan keskelle
\includegraphics[width=0.3\textwidth]{sinetti.png}
\caption{Fysiikan osaston logo, tässä vain mannekiinina kuvan asettelusta.}
\label{fig:sinetti}
% Netistä löytyy useita paketteja jotka auttavat kuvan tai kuvatekstin sijoittelussa.
\end{figure}

\section{Taulukot}

Tuloksia esiteltäessä kannattaa tekstin lisäksi käyttää hyväksi kuvia ja usein myös taulukoita. Kuvateksti tulee kuvan alapuolelle, taulukkoteksti taas taulukon  yläpuolelle.

Kuvien ja taulukoiden tekstit ovat astetta pienemmällä fontilla ja nimiö on lihavoitu.

\begin{table*}
\centering
\caption{Tärkeimmät tulokset}
\label{tab:symbols}
\begin{tabular}{l||l c r} % Pystyviivoja voi laittaa niin monta kuin haluaa, eivätkä ne ole pakollisia taulukon toiminnalle. 
% l-merkintä (left) sijoittaa pylväikön solujen arvot vasemalle
% c-merkintä (center) sijoittaa pylväikön solujen arvot keskelle
% r-merkintä (right) sijoittaa pylväikön solujen arvot oikealle
Koe   & 1 & 2 & 3 \\ 
\hline \hline % \hline luo vaakasuunnassa viivan rivien väliin. Useampi \hline luo useita viivoja kyseisien rivien väliin.ä
$A$   & \num{2.5}  & \num{4.7}  & \num{-11}  \\
$B$   & \num{8.0}  & \num{-3.7} & \num{12.6} \\
$A+B$ & \num{10.5} & \num{1.0}  & \num{1.6}  \\
\hline
%
\end{tabular}
\end{table*}

\chapter{Viittaukset}

Dokumentin viittaustyyleinä ovat ``numeroviitteet'' \texttt{unsrt}, jolloin viitteet tulevat viitelistaan viittausjärjestyksessä,  ja  ``tekijän nimi nimi ja vuosi'' -tyyli \texttt{apalike}, jolloin viitteet tulevat viitelistaan ensimmäisen tekijän sukunimen mukaisessa aakkosjärjestyksessä. Kysy ohjaajaltasi kumpaa viitaustyyliä hän suosittelee käytettäväksi.

Viittauksiin käytetään erillistä .bib-tiedostoa. Tässä dokumentissa se on \texttt{bibliography.bib} ja näyttää tältä:
\begin{verbatim}
@article{einstein,
    author =       "Albert Einstein",
    title =        "{Zur Elektrodynamik bewegter Körper}. ({German})
        [{On} the electrodynamics of moving bodies]",
    journal =      "Annalen der Physik",
    volume =       "322",
    number =       "10",
    pages =        "891--921",
    year =         "1905",
    DOI =          "http://dx.doi.org/10.1002/andp.19053221004"
}
 
@book{latexcompanion,
    author    = "Michel Goossens and Frank Mittelbach and Alexander Samarin",
    title     = "The \LaTeX\ Companion",
    year      = "1993",
    publisher = "Addison-Wesley",
    address   = "Reading, Massachusetts"
}
 
@misc{knuthwebsite,
    author    = "Donald Knuth",
    title     = "Knuth: Computers and Typesetting",
    url       = "http://www-cs-faculty.stanford.edu/\~{}uno/abcde.html"
}
\end{verbatim}

Viittaus tehdään komennolla \texttt{\textbackslash cite\{einstein\}}. Esim. 
\cite{einstein}, \cite{latexcompanion} ja \cite{knuthwebsite}\footnote{Viimeisestä viitteestä puuttuu vuosiluku, koska sitä ei ole määritelty .bib-tiedostossa.}.

\chapter{ Opinnäytetyön rakenne}

Tutustu Kielijelppiin http://www.kielijelppi.fi, joka Helsingin  yliopiston kielikeskuksen ylläpitämä verkko-opas hyvään akateemiseen kielenkäyttöön suomeksi ja ruotsiksi.  

\section{Johdanto}
Johdannossa kuvataan työn tausta, sen liittyminen muuhun tutkimukseen, käytetty tarkastelutapa sekä työn tavoitteet. Ajatuksena on siis kertoa, mihin tieteellisiin kysymyksiin haetaan vastausta ja millä tavoin.

Johdantoon voidaan sisällyttää historiikki ja kirjallisuuskatsaus, jos tämä on tarpeen asian ymmärtämiseksi. Jos kirjallisuuskatsauksesta tulee pitkä, voi kuitenkin olla mielekästä erottaa se omaksi luvukseen.

\section{Tekstiluvut}

Tutkielman lukujako riippuu sen sisällöstä. Mieti, mikä on oman työsi kannalta tarkoituksenmukaista.

Jos työn ymmärtäminen vaatii enemmän teoreettista tietoa kuin  opintojen samassa vaiheessa olevalla mutta työn aiheeseen perehtymättömällä   fysiikan opiskelijalla voi olettaa olevan, teoreettinen tausta yhtälöineen on syytä kuvata omassa luvussa (esim. ``Teoriaa'').
Kielenkäytöissä tulee pitäytyä  asiallisessa kirjakielessä puhekielisiä ilmaisuja välttäen.

Jos työhön sisältyy omia mittauksia tai data-analyysiä, kuvaa ensin käyttämäsi  menetelmät  ja/tai aineisto (luku ``Aineisto ja/tai  menetelmät'') ja sen jälkeen saamasi tulokset (yhdessä tai useammassa luvussa).

Puhtaissa kirjallisuustöissä luvut ``Aineisto ja menetelmät'' ja ``Tulokset'' eivät ole tarpeen. Lukujako valitaan niin, että  asiat  voidaan esitellä mahdollisimman loogisessa järjestyksessä.

\section{Teoriaa}
%
% Tässä esimerkit yhtälöstä sekä lähdeviittauksista.
%
Mahdollisessa teorialuvussa voidaan muun muassa esitellä työssä käytettäviä yhtälöitä. Yleisen omegayhtälön
\begin{equation}
\label{eq:omega}
L(\omega)=F_V + F_T + F_V + F_Q + F_A
\end{equation}
avulla voidaan arvioida ilmakehän pystyliikkeitä. Einsteinille  yhtälö (\ref{eq:omega}) ei kuitenkaan ollut tuttu (\cite{einstein}), eivätkä myöskään \cite{latexcompanion} tai \cite{knuthwebsite} käsittele sitä. 
\nomenclature{$L(\omega)$}{Omega \nomunit{yksikkö}}

\section{Tulokset}
%
% Tässä esimerkit kuvien ja taulukoiden käytöstä.
%
Tuloksia esiteltäessä kannattaa tekstin lisäksi käyttää hyväksi kuvia ja usein myös taulukoita. Muista että kuvateksti tulee kuvan alapuolelle, taulukkoteksti taas taulukon yläpuolelle.

\section{Johtopäätökset}

Kertaa lyhyesti työn tavoitteet ja käyttämäsi menetelmät. Tee yhteenveto tärkeimmistä tuloksista. Pohdi myös tulosten merkitystä ja mahdollisia jatkotutkimustarpeita.

\begin{appendices}
\myappendixtitle

\chapter{Liitteet}
Liitteissä voi esitellä esimerkiksi työssä käytettyjä tietokonekoodeja:
\begin{verbatim}
#!/bin/bash          
text="Hello World!"
echo $text
\end{verbatim}

\end{appendices}

% STEP 5:
% Uncomment the following lines and set your .bib file and desired bibliography style
% to make a bibliography with BibTeX.
% Alternatively you can use the thebibliography environment if you want to add all
% references by hand.

\cleardoublepage %fixes the position of bibliography in bookmarks
\phantomsection

\addcontentsline{toc}{chapter}{\bibname} % This lines adds the bibliography to the ToC
%\bibliographystyle{unsrt} % numbering 
\bibliographystyle{apalike} % name, year
\bibliography{bibliography.bib}

\end{document}
